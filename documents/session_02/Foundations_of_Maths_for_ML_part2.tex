\documentclass[]{article}
\usepackage[utf8]{inputenc}
\usepackage[english]{babel}
\usepackage{csquotes}
\usepackage{graphicx}
\usepackage{minted}
\usepackage{amsmath}
\usepackage{amssymb}
\usepackage{parskip}
\usepackage[
backend=biber,
style=alphabetic,
sorting=ynt
]{biblatex}
\addbibresource{references.bib} 
\graphicspath{ {images/} }

%opening
\title{Foundations of Math for ML, part 2}
\author{Hoang Nhat Minh Tran \thanks{Green Global Information Technology JSC}}
\date{\today}

\begin{document}

\maketitle

% The introduction
\begin{abstract}

\end{abstract}

\section{Linear Algebra}

Now that you can store and manipulate data, let’s briefly review the subset of basic linear algebra that you will need to understand most of the models. We will introduce all the basic concepts, the corresponding mathematical notation, and their realization in code all in one place. If you are already confident in your basic linear algebra, feel free to skim through or skip this chapter.

\begin{minted}[frame=single]{python}
import torch
\end{minted}

\subsection{Scalars}

If you never studied linear algebra or machine learning, you are probably used to working with one number at a time. And know how to do basic things like add them together or multiply them. For example, in PaloAlto, the temperature is 52 degrees Fahrenheit. Formally, we call these values \textit{scalars}. If you wanted to convert this value to Celsius (using metric system’s more sensible unit of temperature measurement), you would evaluate the expression $ f = c \times \frac{9}{5} + 32 $ setting $ f $ to 52. In this equation, each of the terms 32, 5, and 9 is a scalar value. The placeholders $ c $ and $ f $ that we use are called variables and they represent unknown scalar values.

In mathematical notation, we represent scalars with ordinary lower-cased letters $ (x,y,z) $. We also denote the space of all scalars as $ \mathbb{R} $. For expedience, we are going to punt a bit on what precisely a space is, but for now, remember that if you want to say that $ x $ is a scalar, you can simply say $ x \in \mathbb{R} $. The symbol $ \in $ can be pronounced “in” and just denotes membership in a set.

In PyTorch, we work with scalars by creating Tensors with just one element. In this snippet, we instantiate two scalars and perform some familiar arithmetic operations with them, such as addition, multiplication,division and exponentiation.

\begin{minted}[frame=single]{python}
x = torch.tensor([3.0])
y = torch.tensor([2.0])
print(f'x + y = {x+y}')
print(f'x * y = {x*y}')
print(f'x / y = {x/y}')
print(f'x ** y = {torch.pow(x,y)}')
\end{minted}

\begin{minted}[frame=single]{python}
x + y = tensor([5.])
x * y = tensor([6.])
x / y = tensor([1.5000])
x ** y = tensor([9.])
\end{minted}

We can convert any Tensor to a Python float by calling its \textit{numpy} method. Note that this is typically a bad idea. While you are doing this, Tensor has to stop doing anything else in order to hand the result and the process control back to Python. And unfortunately, Python is not very good at doing things in parallel.

So avoid sprinkling this operation liberally throughout your code or your networks will take a long time to train.

\begin{minted}[frame=single]{python}
x.numpy()
\end{minted}

\begin{minted}[frame=single]{python}
array([3.], dtype=float32)
\end{minted}

\subsection{Vectors}

You can think of a vector as simply a list of numbers, for example $ [1.0,3.0,4.0,2.0] $. Each of the numbers in the vector consists of a single scalar value. We call these values the \textit{entries} or \textit{components} of the vector.Often, we are interested in vectors whose values hold some real-world significance. For example, if we are studying the risk that loans default, we might associate each applicant with a vector whose components correspond to their income, length of employment, number of previous defaults, etc. If we were studying the risk of heart attacks hospital patients potentially face, we might represent each patient with a vector whose components capture their most recent vital signs, cholesterol levels, minutes of exercise per day, etc.In math notation, we will usually denote vectors as bold-faced, lower-cased letters (\textbf{u}, \textbf{v}, \textbf{w}). In PyTorch, we work with vectors via 1D Tensors with an arbitrary number of components.

\begin{minted}[frame=single]{python}
x = torch.arange(4)
print(f'x = {x}')
\end{minted}

\begin{minted}[frame=single]{python}
x = tensor([0, 1, 2, 3])
\end{minted}

We can refer to any element of a vector by using a subscript. For example, we can refer to the 4th element of \textbf{u} by $ u_4 $. Note that the element $ u_4 $ is a scalar, so we do not bold-face the font when referring to it. Incode, we access any element $ i $ by indexing into the Tensor.

\begin{minted}[frame=single]{python}
x[3]
\end{minted}

\begin{minted}[frame=single]{python}
tensor(3)
\end{minted}

\subsection{Length, dimensionality and shape}

\subsection{Matrices}

\subsection{Tensors}

\subsection{Basic properties of tensor arithmetic}

\subsection{Sums and means}

\subsection{Dot products}

\subsection{Matrix-vector products}

\subsection{Matrix-matrix multiplication}

\subsection{Norms}

\subsection{Norms and objectives}

\subsection{Intermediate linear algebra}

\subsection{Summary}

In just a few pages (or one Jupyter notebook) we have taught you all the linear algebra you will need to understand a good chunk of neural networks. Of course there is a lot more to linear algebra. And a \textins{lot} of that math \textit{is} useful for machine learning. For example, matrices can be decomposed into factors, and these decompositions can reveal low-dimensional structure in real-world datasets. There are entire sub fields of machine learning that focus on using matrix decompositions and their generalizations to high-order tensors to discover structure in datasets and solve prediction problems. But this book focuses on deep learning.And we believe you will be much more inclined to learn more mathematics once you have gotten your hands dirty deploying useful machine learning models on real datasets. So while we reserve the right to introduce more math much later on, we will wrap up this chapter here.

\section{Recap}

% You just trained your first machine learning model. We saw that by training the model with input data and the corresponding output, the model learned to multiply the input by 1.8 and then add 32 to get the correct result.

\nocite{*}

\printbibliography

\end{document}